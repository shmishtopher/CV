% Copyright 2022 Christopher K. "Shmish" Schmitt
% 
% Licensed under the Apache License, Version 2.0 (the "License");
% you may not use this file except in compliance with the License.
% You may obtain a copy of the License at
% 
%     http://www.apache.org/licenses/LICENSE-2.0
% 
% Unless required by applicable law or agreed to in writing, software
% distributed under the License is distributed on an "AS IS" BASIS,
% WITHOUT WARRANTIES OR CONDITIONS OF ANY KIND, either express or implied.
% See the License for the specific language governing permissions and
% limitations under the License.

\documentclass[a4paper]{article}


\usepackage{fullpage}
\usepackage{hyperref}
\usepackage{geometry}
\usepackage{paralist}


\hypersetup{
    colorlinks=false,
    linkcolor=blue,
    filecolor=magenta,      
    urlcolor=cyan,
    pdftitle={Christopher Schmitt - CV},
    pdfpagemode=FullScreen,
}

\geometry{
    a4paper,
    total={170mm,260mm},
    left=20mm,
    top=20mm,
}

\urlstyle{same}
\pagestyle{empty}


% Custom section environment
% #1 - Section name
\newenvironment{cvsection}[1]{%
    \noindent
    \textsc{#1}
    \vspace{4pt}
    \hrule
    \vspace{4pt}
}{\vspace{2pt}}


% University macro
% #1 - University
% #2 - Location
% #3 - Degree
% #4 - GPA
% #5 - Dates
% #6 - Minors
\newcommand{\school}[5] {
    \noindent\textbf{#1}\hfill#2\\
    \noindent#3\hfill#4\\
    \noindent\vspace{8pt}Minors in #5\\
}


% Employer macro
% #1 - Employer
% #2 - Location
% #3 - Role
% #4 - Dates
% #5 - List of responsibilities
\newcommand{\employer}[5] {
    \noindent\textbf{#1}\hfill#2\\
    \noindent\textit{#3}\hfill#4
    \begin{compactitem}
        #5
    \end{compactitem}
    \vspace{16pt}
}


% Project macro
% #1 - Name
% #2 - Technologies
% #3 - Link
% #4 - Description
% \noindent\textbf{#1} {\sl #2} \hfill \ifx&#3&\else \href{#3}{#3}\fi\\
% \noindent#4\\
% \vspace*{50pt}
\newcommand{\project}[4]{
    \noindent\textbf{#1} --- \textit{#2}\hfill\ifx&#3&\else\href{#3}{#3}\fi\\
    \noindent#4\vspace{8pt}\\
}


% Research macro
% #1 - Name
% #2 - Institution
% #3 - Dates
% #4 - Description
\newcommand{\research}[4]{
    \noindent\textbf{#1}\hfill\textit{#2} #3\\
    \noindent#4\vspace{8pt}\\
} 


\begin{document}
    % Document header with name, email, phone number, and portfolio link.
    \begin{center}
        {\huge Christopher K. Schmitt}\\
        \vspace{8pt}
        me@shmish.dev $\cdot$ 8605787614 $\cdot$ \href{https://shmish.dev/}{https://shmish.dev/}
    \end{center}

    % List relevant work experience and previous employers, including location,
    % dates of employment, and role.  List responsibilities in a bulleted list.
    \begin{cvsection}{Experience}
        \employer{PRESCO, Inc.}{Woodbridge, CT}{Software Engineer}{Nov 2023 --- Present}{
            \item Developed firmware for a number of mission-critical embedded systems in the biomed, defense, and consumer domains.
            \item Authored full-stack, bluetooth-enabled mobile applications.
            \item Interfaced directly with clients to determine application needs and requirements.
        }
    
        \employer{BioXcel Therapeutics}{New Haven, CT}{Software Data Engineer}{May 2022 --- Nov 2023}{
            \item Implemented machine learning systems for predicting target binding affinity for potential compounds.
            \item Developed tools for constructing knowledge graphs in the drug re-purposing domain.
            \item Used a variety of AI techniques for link prediction and compound property prediction.
        }

        \employer{TheCoderSchool}{Farmington, CT}{Instructor}{Oct 2018 --- May 2022}{
            \item Taught computer programming and computer science concepts
            \item Developed curricula for teaching foundational concepts in computing and robotics
        }
    \end{cvsection}

    % List relevant university education, including dates, locations, degree 
    % programs, and performance.
    \begin{cvsection}{Education}
        \school{Central Connecticut State University}{New Britain, CT}{BS Computer Science -- Honors}{Sep 2018 --- Jun 2022}{Mathematics, History, and Psychology}
    \end{cvsection}

    % List proficient skills in a table, categorized by field of interest.
    % Include languages as well as major frameworks.
    \begin{cvsection}{Skills}
        \begin{tabular}{ l l }
            Tools:                      & Git, GH Actions, GCP\\
            Systems Programming:        & Rust, C, C++\\
            Machine Learning:           & Python, Jax, PyTorch, TensorFlow\\
            Fullstack Development:      & JavaScript, TypeScript, React\\
            Databases:                  & PostgreSQL, MySQL, Neo4J\\
            MCUs:                       & STM32, ESP32, AVR\\
            Other:                      & Haskell, LaTeX, Java\\
        \end{tabular}
        \vspace{16pt}
    \end{cvsection}

    % List impressive or useful projects, including skill tags, links (if 
    % available), and a description.
    \begin{cvsection}{Projects}
        \project{Twitter-RNN}{TensorFlow, JavaScript}{https://github.com/shmishtopher/Twitter-RNN}{An artificial neural network leveraging BEAM search to generate Tweets indistinguishable to those composed by humans.}
        \project{CoinBlock}{JavaScript}{https://github.com/shmishtopher/CoinBlock}{An extension for detecting and blocking browser-based crypto mining attacks with thousands of active users.}
        \project{pneumonia-CNN}{TensorFlow, JavaScript}{https://github.com/shmishtopher/pneumonia-CNN}{A deep convolutional network for diagnosing pneumonia with a high degree of accuracy.}
    \end{cvsection}

    % List research accomplishments, including the name of the institution,
    % project dates, and project description
    \begin{cvsection}{Research}
        \research{Dark Web Text Classification with RNNs}{CCSU}{2021 --- 2022}{Lead investigator studying and developing unsupervised text classification techniques for analyzing dark web documents.  CCSCNE 2022 Finalist.}
        \research{De Bruijn Graph Genome Assembly Acceleration}{CCSU}{2019 --- 2020}{Lead investigator studying optimal k-mer length for probabilistic genome Assembly using De Bruijn graphs. Submitted to the Central Undergraduate Research Conference.}
    \end{cvsection}
\end{document}
